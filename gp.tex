\let\negmedspace\undefined
\let\negthickspace\undefined
\documentclass[journal,12pt,twocolumn]{IEEEtran}
\usepackage{cite}
\usepackage{amsmath,amssymb,amsfonts,amsthm}
\usepackage{algorithmic}
\usepackage{graphicx}
\usepackage{textcomp}
\usepackage{xcolor}
\usepackage{txfonts}
\usepackage{listings}
\usepackage{enumitem}
\usepackage{mathtools}
\usepackage{gensymb}
\usepackage{comment}
\usepackage[breaklinks=true]{hyperref}
\usepackage{tkz-euclide} 
\usepackage{listings}
\usepackage{gvv}                                        
\def\inputGnumericTable{}                                 
\usepackage[latin1]{inputenc}                                
\usepackage{color}                                            
\usepackage{array}                                            
\usepackage{longtable}                                       
\usepackage{calc}                                             
\usepackage{multirow}                                         
\usepackage{hhline}                                           
\usepackage{ifthen}                                           
\usepackage{lscape}

\newtheorem{theorem}{Theorem}[section]
\newtheorem{problem}{Problem}
\newtheorem{proposition}{Proposition}[section]
\newtheorem{lemma}{Lemma}[section]
\newtheorem{corollary}[theorem]{Corollary}
\newtheorem{example}{Example}[section]
\newtheorem{definition}[problem]{Definition}
\newcommand{\BEQA}{\begin{eqnarray}}
\newcommand{\EEQA}{\end{eqnarray}}
\newcommand{\define}{\stackrel{\triangle}{=}}
\theoremstyle{remark}
\newtheorem{rem}{Remark}

\begin{document}
\bibliographystyle{IEEEtran}

\vspace{3cm}

\title{}
\author{EE23BTECH11047 - Deepakreddy P
}
\maketitle
\newpage
\bigskip

\noindent \textbf{17} \quad 
If a, b, c, d are in G.P, prove that 
$ (a^{n} + b^{n}),(b^{n} + c^{n}),(c^{n} + d^{n}) $ are in G.P and find the Z transform of General term of G.P.\\
\solution

\begin{center}
    \begin{table}[ht]
        \setlength{\arrayrulewidth}{0.3mm}
\setlength{\tabcolsep}{12pt}
\renewcommand{\arraystretch}{1.3}


\begin{center}
\caption{Input Parameters}
\begin{tabular}{ |p{2.0cm}|p{2.0cm}| }

\hline
 {Symbol}&{Input value}\\
\hline
$x(0)$ & $ar^0$ \\
\hline
$x(1) $ & $ar^1$ \\
\hline
$x(2)$ & $ar^2$ \\
\hline
$x(3)$ & $ar^3$ \\
\hline

\end{tabular}
\end{center}

    \end{table}
\end{center}

\begin{align} \implies \frac{\brak{b^n + c^n}}{\brak{a^n + b^n}} &= \frac{\brak{c^n + d^n}}{\brak{b^n + c^n}} \end{align}

\begin{align} \implies \frac{\brak{ar^1}^n + \brak{ar^2}^n}{\brak{ar^0}^n + \brak{ar^1}^n}  &= \frac{\brak{ar^2}^n + \brak{ar^3}^n}{\brak{ar^1}^n + \brak{ar^2}^n} \end{align}

\begin{align} \implies \frac{a^n r^n \brak{1 + r^n}}{a^n \brak{1 + r^n}} &= \frac{a^n r^{2n} \brak{1 + r^n}}{a^n r^n \brak{1 + r^n}} \end{align}

\begin{align} \implies r^n &= r^n \end{align}
Hence proved they are in in G.P
\begin{align}
    x\brak{n} &= x\brak{0}r^{n} u\brak{n}\\
    X\brak{z} &= \frac{x\brak{0}}{1-rz^{-1}}, \quad |z|>|r|\\
    &= \frac{a}{1-rz^{-1}}, \quad |z|>|r|
\end{align}





\end{document}

