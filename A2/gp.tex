\let\negmedspace\undefined
\let\negthickspace\undefined
\documentclass[journal,12pt,twocolumn]{IEEEtran}
\usepackage{cite}
\usepackage{amsmath,amssymb,amsfonts,amsthm}
\usepackage{algorithmic}
\usepackage{graphicx}
\usepackage{textcomp}
\usepackage{xcolor}
\usepackage{txfonts}
\usepackage{listings}
\usepackage{enumitem}
\usepackage{mathtools}
\usepackage{gensymb}
\usepackage{comment}
\usepackage[breaklinks=true]{hyperref}
\usepackage{tkz-euclide} 
\usepackage{listings}
\usepackage{gvv}                                        
\def\inputGnumericTable{}                                 
\usepackage[latin1]{inputenc}                                
\usepackage{color}                                            
\usepackage{array}                                            
\usepackage{longtable}                                       
\usepackage{calc}                                             
\usepackage{multirow}                                         
\usepackage{hhline}                                           
\usepackage{ifthen}                                           
\usepackage{lscape}

\newtheorem{theorem}{Theorem}[section]
\newtheorem{problem}{Problem}
\newtheorem{proposition}{Proposition}[section]
\newtheorem{lemma}{Lemma}[section]
\newtheorem{corollary}[theorem]{Corollary}
\newtheorem{example}{Example}[section]
\newtheorem{definition}[problem]{Definition}
\newcommand{\BEQA}{\begin{eqnarray}}
\newcommand{\EEQA}{\end{eqnarray}}
\newcommand{\define}{\stackrel{\triangle}{=}}
\theoremstyle{remark}
\newtheorem{rem}{Remark}

\begin{document}
\bibliographystyle{IEEEtran}

\vspace{3cm}

\title{}
\author{EE23BTECH11047 - Deepakreddy P
}
\maketitle
\newpage
\bigskip

\noindent \textbf{17} \quad 
If a, b, c, d are in G.P, prove that 
$ (a^{n} + b^{n}),(b^{n} + c^{n}),(c^{n} + d^{n}) $ are in G.P.\\
\solution

\begin{align}   
a &= ar^{0}\\
b &= ar^{1}\\
c &= ar^{2}\\
d &= ar^{3}
\end{align}

\bigskip
\begin{align} \frac{(b^n + c^n)}{(a^n + b^n)} &= \frac{(c^n + d^n)}{(b^n + c^n)} \end{align}

\begin{align} \frac{(ar^1)^n + (ar^2)^n}{(ar^0)^n + (ar^1)^n}  &= \frac{(ar^2)^n + (ar^3)^n}{(ar^1)^n + (ar^2)^n} \end{align}

\begin{align} \frac{a^n r^n(1 + r^n)}{a^n(1 + r^n)} &= \frac{a^n r^{2n}(1 + r^n)}{a^n r^n (1 + r^n)} \end{align}

\begin{align} r^n = r^n \end{align}
\bigskip
Hence proved they are in in G.P
\end{document}

